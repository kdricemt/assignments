\documentclass[15pt,uplatex,dvipdfmx]{jsarticle}
\usepackage{indentfirst}
\usepackage{tabularx}
\usepackage[hmargin=2cm,vmargin=2cm]{geometry}
\usepackage{amsmath}

%%% tabularx環境用マクロ
%%セル内で中央揃えをする。
\newcolumntype{C}{>{\centering\arraybackslash}X}
%%セル内で右揃え。
\newcolumntype{R}{>{\raggedright\arraybackslash}X}
%%セル内で左揃え。
\newcolumntype{L}{>{\raggedleft\arraybackslash}X}

\begin{document}
\setcounter{section}{3}
\section{アポジモーター、軌道用制御燃料の推算}

\subsection{仮定}
\begin{enumerate}
  \item 構造系質量は機器重量の総和の10\%とする。
  \item 計装、配線重量は機器重量の総和の7\%とする。
  \item システムマージンとして上記合計の7\%をとる。
  \item 姿勢制御用$\Delta$Vは10m
  \item 姿勢制御用$\Delta$Vマージンは5\%
  \item Isp:起動姿勢制御用(ヒドラジン) 170s、アポジモーター(個体) 280s
  \item タンク重量は燃料重量の10\%と仮定する。
  \item マージン(2\%)も含めてGTO投入時の総重量を求める
\end{enumerate}

\subsection{ドライ重量の計算}
 仮定より、以下のようになる
 \begin{align*}
   タンク重量 & : W_T = 0.1W_{prop} \\
   機器重量 & : W_E = 810[kg] + W_T \\
   構造重量 & : W_S = 0.1W_E \\
   計装 \cdot 配線重量 & : W_W = 0.07W_E \\
   ドライ質量 & : W_D = 1.07(W_E + W_S + W_W) = 1.07 \times 1.17 W_E \\
 \end{align*}

 \subsection{燃料重量の計算}
 燃料重量を$M_{prop}$とすると
 \begin{equation}
   M_{prop} = M_{dry} \exp ( \Delta V/gIsp - 1)
   \Leftrightarrow \Delta V = gIsp \ln{\frac{W_{dry} + W_{prop}}{W_{dry}}}
 \end{equation}
姿勢軌道制御用、アポジモーターとして必要な$\Delta V$をそれぞれ$\Delta V_{sk}、\Delta V_{ap}$
とすると、1章より、
\begin{equation}
  \begin{cases}
    \Delta V_{sk} = 1.05(\Delta V_{NS} + \Delta V_{ES} + 10[m/s]) = 399.06[m/s] \\
    \Delta V_{ap} = 1869.1[m/s]
  \end{cases}
\end{equation}
(1)より、
\begin{equation}
  \begin{cases}
    M_{skfuel} = (W_D) \exp ( \Delta V_{sk}/gIspsk - 1) \\
    M_{apfuel} = (W_D + M_opfuel)(\exp ( \Delta V_{ap}/gIspap - 1))
  \end{cases}
\end{equation}
これを解くと、
\begin{equation}
  \begin{cases}
    M_{skfuel} = 338.48[kg] \\
    M_{apfuel} = 1551.2[kg]
  \end{cases}
\end{equation}
となる。

\subsection{タンク重量 $\cdot$ 体積の計算}
タンク重量$M_T$は仮定より、
\begin{equation}
  \begin{cases}
    M_{T_{sk}} = 0.1M_{skfuel} = 33.85[kg] = 16.924 \times 2 \\
    M_{T_{ap}} = 0.1M_{apfuel} = 155.12[kg]
  \end{cases}
\end{equation}
ヒドラジンの密度$\rho_{sk}$と、個体燃料(過塩素酸アンモニウムと仮定)として、その
密度$\rho_{ap}$は、
\begin{equation}
  \begin{cases}
    rho_{sk} = 1011[kg/m^3] \\
    rho_{ap} = 1950[kg/m^3]
  \end{cases}
\end{equation}
である。それぞれのタンク体積Vは、
\begin{equation}
  \begin{cases}
    V_{sk} = \frac{M_{skfuel}}{\rho sk} = \frac{338.48[kg]}{1011[kg/m^3]}
     \approx 0.3348[m^3] \\
     V_{ap} = \frac{M_{apfuel}}{\rho ap} = \frac{1551.2[kg]}{1950[kg/m^3]}
      \approx 0.7955[m^3] \\
  \end{cases}
\end{equation}
球形タンクの半径rは、
\begin{equation}
  \begin{cases}
    r_{sk} > \sqrt[3]{\frac{3\frac{V_{sk}}{2}}{4\pi}} \approx 0.3419[m] \\
    r_{ap} > \sqrt[3]{\frac{3V_{ap}}{4\pi}} \approx 0.5748[m^3]
  \end{cases}
\end{equation}
を満たさなければならないから、
\begin{equation}
  \begin{cases}
    r_{sk} = 35[cm] \\
    r_{ap} = 58[cm]
  \end{cases}
\end{equation}

\subsection{打ち上げ重量の計算}
以上より打ち上げ重量$W_{pL}$は、
\begin{align*}
  W_{PL} & = 1.02(W_D + M_skfuel + M_apfuel) \\
         & \approx 3203.1[kg]
\end{align*}
となる





\setcounter{section}{1}

\end{document}
