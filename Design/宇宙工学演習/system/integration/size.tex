\section{衛星寸法の決定}

\subsection{仮定}
\begin{enumerate}
  \item 実装効率は50\%以下とする
  \item アポジモーターはスラストチューブ内部に配置する
  \item ヒドラジンタンクはバルクヘッド中央に配置する
\end{enumerate}

\subsection{構造内部の機器の体積と衛星体積の概算}
構造内部の機器の体積は、下表のようになる\par
\setlength{\parskip}{1.0\baselineskip}

\setlength{\tabcolsep}{.5zw}
\begin{tabularx}{40zw}{|C|C C C|} \hline
  機器名 & 寸法[cm] & 数量 & $体積[cm^3]$ \\ \hline
  Kaバンド中継機 & $138 \times 70 \times 20$ & 1 & 193,200 \\
  Sバンド中継機 & $70 \times 70 \times 20$ & 1 & 98,000 \\
  IRU & $30 \times 38 \times 30$ & 1 & 34,200 \\
  AOCE & $20 \times 15 \times 7$ & 1 & 2,100 \\
  リアクションリアクションホイール & $30 \times 30 \times 10$ & 1 & 9,000 \\
  TT \& Cユニット & $80 \times 60 \times 20$ & 1 & 96,000 \\
  オンボード計算機 & $40 \times 26 \times 12$ & 1 & 12,480 \\
  パドル駆動モータ & $19 \times 20 \times 34$ & 2 & 25,840 \\
  バッテリ & $35 \times 25 \times 20$ & 2 & 35,000 \\
  電源制御部 & $20 \times 30 \times 20$ & 2 & 24,000 \\
  ヒドラジンタンク & r=35(球) & 2 & 359,189 \\
  アポジタンク & r=58(球) & 1 & 817,283 \\ \hline
  \multicolumn{3}{|c}{合計} & 1,706,292 \\ \hline
\end{tabularx} \par
したがって衛星体積の最小値$V_{min}$は、
\begin{equation}
  V_{min} = \frac{1706292}{0.5} = 3412584[cm^3]
\end{equation}
となるため、一辺の長さはおよそ
\begin{equation}
  a = sqrt[3]{3412584} \approx 150[cm]
\end{equation}
となる

\subsection{衛星寸法の決定$\cdot$搭載面積の確認}
円形のフェアリングに収容することを考え、$\pm$TAR面は正方形として設計した。
ヒドラジンタンクをバルクヘッド中央、アポジタンクをスラストチューブ内に配置する関係から、
一辺の長さdは、
\begin{equation}
  d > 2(2r_{sk} + r_{ap}) = 2(2 \times 35[cm] + 58[cm]) = 256[cm]
\end{equation}
を必ず満たさなくてはならない。中継機を(水平方向)中央付近に配置することを考え,
d=320[cm]とし、また中継機の高さを考えて、h=150[cm]とした。
5-1より、これは条件を満たすと考えられる。
