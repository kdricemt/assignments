\section{ミッション概要}
静止軌道の人工衛星を設計する

\subsection{軌道}
軌道傾斜角\ 0度 \par
東経135度上空\ 静止軌道\ (半径42160kmの静止軌道)

\subsection{寿命}
7年

\subsection{ミッション機器}
図1-1参照。搭載面要求として、アンテナ、アンテナタワーは地球指向面上とする\par\medskip
\begin{table}[H]
  \caption{ミッション機器一覧}
\begin{tabular}{|c|c|c|c|c|} \hline
  \multicolumn{2}{|c|}{} & 重量[kg] & 使用電力[W] & 許容温度[℃] \\ \hline
   & \o0.7m\ パラボラアンテナ(S バンド) & 5 & 0 & 10-40 \\ \cline{2-5}
   \raisebox{.5\normalbaselineskip}[0pt][0pt]{uplink}
   & \o1.5m\ パラボラアンテナ(Ka バンド) & 23 & 0 & 10-40 \\ \hline
   & \o0.8m\ パラボラアンテナ(S バンド) & 6 & 0 & 10-40 \\ \cline{2-5}
   \raisebox{.5\normalbaselineskip}[0pt][0pt]{downlink}
   & \o1.6m\ パラボラアンテナ(Ka バンド) & 26 & 0 & 10-40 \\ \hline
   \multicolumn{2}{|c|}{アンテナタワー} & 70 & 0 & -45-65 \\ \hline
   \multicolumn{2}{|c|}{Kaバンド中継機($1380\times700\times200mm$)} & 180 & 867 & 5-40 \\ \hline
   \multicolumn{2}{|c|}{Sバンド中継機($700\times700\times200mm$)} & 60 & 330 & 5-40 \\ \hline
\end{tabular}\par\medskip
\end{table}
\newpage

\section{$\Delta$Vの見積もり}
\subsection{概要}
種子島(緯度30度)から半径$R_{PO}=6600km$のパーキング軌道まで入れ、そこから半径$R_{GEO}=42160km$のGEOへの
トランスファ軌道へ入れる。トランスファ軌道に入るところまではロケットの責任とする。
アポジ点で衛星搭載のキックモーターをふかす際に必要な$\Delta$Vを見積もる。
\subsection{アポジ点での$\Delta$V}
以下近地点における諸元の添え字をPとし、遠地点における諸元の添え字をAとする。
アポジ点でふかすキックモーターの$\Delta$Vを${\Delta}V_A$とし、GTOでの速度$V_{GTO}$を考える。
ケプラー第二法則より、
\begin{equation}
  R_{GEO}V_{GTO_A}=R_{PO}V_{GTO_P}
\end{equation}
また、エネルギー保存則より、
\begin{equation}
  \frac{1}{2} V_{GTOA}^2 - \frac{\mu}{R_{GEO}}
  =  \frac{1}{2} V_{GTOP}^2 - \frac{\mu}{R_{PO}}
\end{equation}
以上より,
\begin{align*}
  V_{GTOA} & = \sqrt{\frac{2\mu R_{PO}}{R_{GEO}(R_{PO}+R_{GEO})}} \\
           & = \sqrt{\frac{2\times3.986\times10^{14}[m^3/s^2]\times6600\times10^3[m]}
           {42160\times10^3[m]\times(6600\times10^3[m]+42160\times10^3[m])}} \\
           & = 1599.83...[m/s] \\
           & \approx 1599.8[m/s]
\end{align*}
となる。また、GEOでの速度$V_{GEO}$は、
\begin{align*}
  V_{GEO} &= \sqrt{\frac{\mu}{R_{GEO}}} \\
          &= \sqrt{\frac{3.986\times10^14[m^3/s^2]}{42160\times10^3[m]}} \\
          &= 3074.81...[m/s] \\
          & \approx 3074.8[m/s]
\end{align*}
である。よって速度三角形より、
\begin{equation}
  {\Delta}V_A = \sqrt{V_{GEO}^2+V_{GTOA}^2-2V_{GEO}V_{GTOA}\cos{30^ \circ }}
              = 1869.14[m/s]
               \approx{1869.1}[m/s]
\end{equation}
となる。
\subsection{軌道の維持に必要な$\Delta V$}
次に軌道の維持に必要な$\Delta V_{orbit}$を求める。\par
軌道面と月軌道、黄道のなす角をそれぞれ$\alpha$、$\gamma$とすると、月と太陽による軌道傾斜角
方向のずれは、$i=0^{orbit}$の静止軌道上で、
\begin{equation}
  \begin{cases}
    \Delta V_{MOON} = 102.67 \cos \alpha \sin \alpha [m/s \cdot year] \approx 36.93[m/s \cdot year] \\
    \Delta V_{SUN} = 40.17 \cos \gamma \sin \gamma [m/s \cdot year] \approx 14.45[m/s \cdot year]
  \end{cases}
\end{equation}
となる。また衛星は安定点である東経75度、225度に向かってドリフトしていく。今回の静止衛星は東経135度上に
静止するので、東経75度に向かってドリフトする。これによるずれは、
\begin{equation}
\Delta V_D = 1.715 \sin \bigl\{ 2 \times (135-75)[^ \circ] \bigr\} \approx 1.485[m/s \cdot year]
\end{equation}
である。よって衛星に必要な南北、東西方向の$\Delta V_{orbit}$は、
\begin{equation}
  \begin{cases}
    \Delta V_{NS} = 7[year] \times (\Delta V_{MOON} + \Delta V_{SUN}) \approx 359.7[m/s \cdot year] \\
    \Delta V_{EW} = 7[year] \times \Delta V_D \approx 10.40[m/s \cdot year] \\
  \end{cases}
\end{equation}
