\documentclass[class=article, crop=false, dvipdfmx, fleqn]{standalone}
\title{航空機設計法第一 \\
レポート課題3 \ 機体三面図(初期案)}
\author{学籍番号 03-170313 飯山 敬大\\
        }
\date{\today}

% packages and libraries
\usepackage[utf8]{inputenc}				%fonts
\usepackage[ipaex]{pxchfon}
\usepackage{pifont}
\usepackage{mathtools, amssymb, mathrsfs, bbm,nccmath}	%math
\usepackage{siunitx, physics}
\usepackage[table]{xcolor}				%colors
\usepackage{tabularx}
\usepackage[dvipdfmx]{graphicx}					%figures
\usepackage{subcaption, wrapfig}
\usepackage{tikz}
\usetikzlibrary{calc, patterns, decorations, angles, calendar, backgrounds, shadows, mindmap}
\usepackage{tcolorbox}					%tables
\usepackage{longtable, float, multirow, array, listliketab, enumitem, tabularx}
\usepackage{listings}					%listings
\usepackage{comment}
\usepackage{hyperref}					%URL, link
\usepackage{url}
\usepackage{pxjahyper}
\usepackage{overcite}					%setting of citation
\usepackage{pxrubrica}					%rubi
\usepackage{fancyhdr, lastpage}			%pagelayout
\usepackage{import, grffile}			%file management
\usepackage{standalone}
\usepackage{bm}
\usepackage{empheq}
\usepackage{pdfpages}
\usepackage{multicol}
% set up for siunitx
\sisetup{%
	%detect-family = true,
	detect-inline-family = math,
	detect-weight = true,
	detect-inline-weight = math,
    %input-product = *,
    quotient-mode = fraction,
	fraction-function = \frac,
	inter-unit-product = \ensuremath{\hspace{-1.5pt}\cdot\hspace{-1.5pt}},
	per-mode = symbol,
	product-units = single,
	}

% setting of line skip
\setlength{\lineskiplimit}{6pt}
\setlength{\lineskip}{6pt}

% setting of indent
\setlength{\parindent}{1zw}
\setlength{\mathindent}{5zw}

% change cite form
\renewcommand{\citeform}[1]{[#1]}

% number equations only when they are referred to in the text
\mathtoolsset{showonlyrefs=true}
%\graphicspath{{images/}{../images/}}

% set up for hyperref
\hypersetup{%
	bookmarksnumbered = true,%
	hidelinks,%
	colorlinks = true,%
	linkcolor = black,%
	urlcolor = cyan,%
	citecolor = black,%
	filecolor = magenta,%
	setpagesize = false,%
	}

\pdfstringdefDisableCommands{%
\renewcommand*{\bm}[1]{#1}%
% any other necessary redefinitions
}
% Include \subsubsection in ToC
\setcounter{tocdepth}{3}

% tabularx
\newcolumntype{C}{>{\centering\arraybackslash}X} %セル内で中央揃え
\newcolumntype{R}{>{\raggedright\arraybackslash}X} %セル内で右揃え
\newcolumntype{L}{>{\raggedleft\arraybackslash}X}

\begin{document}
\section{重量推算}
\subsection{主翼重量}
主翼重量$W_w$を考える. 終局荷重倍数$N_{ult}$と零燃料時の最大質量$W_{mzf}$及び
50$\%$翼弦長での後退角$\Lambda_{1/2}$を,
\begin{align}
  N_{ult} &= 3.8 \\
  W_{mzf} &= W_{OE} + W_{PL} = W_{TO} - W_{F} = 658000 [Ib] \\
  \Lambda_{1/2} &= tan^{-1}
  \biggl( \frac{\frac{b}{2}tan\Lambda + \frac{c_t}{4} + \frac{c_r}{4}}{\frac{b}{2}} \biggr)
  = 27.4 [deg]
\end{align}
とする. また主翼濡れ面積を, キンク部分を考慮して,
\begin{equation}
  S_w = S * 1.2 = 11697 [ft^2]
\end{equation}
とすると,
\begin{align}
  \hspace{-5mm} W_w &= 0.0017 \times W_{mzf} {\biggl( \frac{b}{cos\Lambda_{1/2}} \biggr)}^{0.75}
  \biggl \{ 1 + {\biggl( \frac{6.25}{b}cos\Lambda_{1/2} \biggr)}^{0.5}\biggr \}
  {N_{ult}}^{0.55} { \biggl( \frac{bS_w}{t_{root}c_rW_{mzf}cos\Lambda_{1/2}}\biggr)}^{0.30} \\
  &= 157450 [Ib]
\end{align}
となる.

\subsection{尾翼重量}
今回は尾翼を考えないので, 省略する.

\subsection{胴体質量}
胴体濡れ面積は, 胴体を高さ12.2[ft],幅65.5[ft],長さ45[ft]の角柱と近似する(第4章参照)ことにより,
\begin{equation}
  S_{fus} \approx  2 * (12.2 \times 65.5  + 12.2 \times 45 + 65.5 \times 45)= 8591 [ft^2]
\end{equation}
また, 設計急降下速度$V_d$は,
\begin{equation}
  V_d = (0.80 + 0.05) \times 0.867 \times 111 = 822[ft/s]
\end{equation}
となる. 以上から$W_{fus}$は,
\begin{equation}
  W_{fus} = 0.0065 \times V_d^{0.5} \times 1.85 \times {S_{fus}}^{1.2}
  = 18129 [Ib]
\end{equation}
ただし体積計算にあたっては胴体の最大幅, 最大高さの値で計算したので、実際にはより
小さくなると考えられる.

\subsection{ナセル重量}
今回採用したエンジンは高バイパス比エンジンなので,
\begin{equation}
  F_{nacelle} = 0.065
\end{equation}
ここで, 第二回レポートより
\begin{equation}
  T_{TO} = 244044 [Ib]
\end{equation}
であるから,ナセル重量$W_n$は,
\begin{equation}
  W_n = F_{nacelle}T_{TO} = 15862 [Ib]
\end{equation}

\subsection{脚重量}
今回は低翼機として,
\begin{equation}
  K_{gr} = 1.0
\end{equation}
とすると、
\begin{align}
  W_{mg} &= K_{gr}(40.0 + 0.16 {W_{TO}}^{0.75} + 0.019W_{TO} + 1.5 \times 10^{-5}{W_{TO}}^{1.5})
  = 34445 [Ib]\\
  W_{ng} &= K_{gr}(20.0 + 0.10 {W_{TO}}^{0.75} + 2.0 \times 10^{-6}{W_{TO}}^{1.5})
  = 4628 [Ib]
\end{align}
より,
\begin{equation}
  W_g = W_{mg} + W_{ng} = 39074 [Ib]
\end{equation}

\subsection{推進系統重量}
エンジンは第4章を参考にして,
\begin{equation}
  W_{eng} = 15596 \times 3 = 46788 [Ib]
\end{equation}
となる.よって, 教科書(9.11)式より,
\begin{equation}
  W_P = 1.16W_{eng} + 5950 = 60224 [Ib]
\end{equation}

\subsection{装備品重量}
遠距離機体なので,
\begin{equation}
  F_{fix} = 0.08
\end{equation}
とすると, 装備重量$W_{fix}$は,
\begin{equation}
  W_{fix} = F_{fix}W_{TO} = 71680 [Ib]
\end{equation}
となる.

\subsection{運用に必要なアイテム重量}
長距離路線なので, $F_{OP} = 35$として, 装備重量$W_{OP}$は,
\begin{equation}
  W_{OP} = 187N_{crew} + F_{OP}P = 187 \times 17 + 35 \times 420 = 17879 [Ib]
\end{equation}
となる.

\subsection{運用空虚重量}
以上を合計し, $W_{tfo}$は最大離陸重量の$0.5\%$として,
\begin{align}
  W_{ME} &= W_w + W_{fus} + W_{n} + W_{g} \\
  W_{OE} &= W_{ME} + W_p + W_{fix} + W_{tfo} + W_{crew} + W_{OP} = 388265 [Ib]
\end{align}
となった.レポート2での計算の結果は,
\begin{equation}
  W_{OE} = 497787 [Ibs]
\end{equation}
であったから, かなりの誤差が生じてしまっているが, BWB機に特化した統計データがなく
胴体や主翼の計算が精度を欠いていることを考えると, 今後のさらに詳細なサイジングにより, この差を
埋めていくものとする.



\end{document}
