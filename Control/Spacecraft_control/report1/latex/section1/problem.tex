\documentclass[class=article, crop=false, dvipdfmx, fleqn]{standalone}
\title{航空機設計法第一 \\
レポート課題3 \ 機体三面図(初期案)}
\author{学籍番号 03-170313 飯山 敬大\\
        }
\date{\today}

% packages and libraries
\usepackage[utf8]{inputenc}				%fonts
\usepackage[ipaex]{pxchfon}
\usepackage{pifont}
\usepackage{mathtools, amssymb, mathrsfs, bbm,nccmath}	%math
\usepackage{siunitx, physics}
\usepackage[table]{xcolor}				%colors
\usepackage{tabularx}
\usepackage[dvipdfmx]{graphicx}					%figures
\usepackage{subcaption, wrapfig}
\usepackage{tikz}
\usetikzlibrary{calc, patterns, decorations, angles, calendar, backgrounds, shadows, mindmap}
\usepackage{tcolorbox}					%tables
\usepackage{longtable, float, multirow, array, listliketab, enumitem, tabularx}
\usepackage{listings}					%listings
\usepackage{comment}
\usepackage{hyperref}					%URL, link
\usepackage{url}
\usepackage{pxjahyper}
\usepackage{overcite}					%setting of citation
\usepackage{pxrubrica}					%rubi
\usepackage{fancyhdr, lastpage}			%pagelayout
\usepackage{import, grffile}			%file management
\usepackage{standalone}
\usepackage{bm}
\usepackage{empheq}
\usepackage{pdfpages}
\usepackage{multicol}
% set up for siunitx
\sisetup{%
	%detect-family = true,
	detect-inline-family = math,
	detect-weight = true,
	detect-inline-weight = math,
    %input-product = *,
    quotient-mode = fraction,
	fraction-function = \frac,
	inter-unit-product = \ensuremath{\hspace{-1.5pt}\cdot\hspace{-1.5pt}},
	per-mode = symbol,
	product-units = single,
	}

% setting of line skip
\setlength{\lineskiplimit}{6pt}
\setlength{\lineskip}{6pt}

% setting of indent
\setlength{\parindent}{1zw}
\setlength{\mathindent}{5zw}

% change cite form
\renewcommand{\citeform}[1]{[#1]}

% number equations only when they are referred to in the text
%\mathtoolsset{showonlyrefs=true}
%\graphicspath{{images/}{../images/}}

% set up for hyperref
\hypersetup{%
	bookmarksnumbered = true,%
	hidelinks,%
	colorlinks = true,%
	linkcolor = black,%
	urlcolor = cyan,%
	citecolor = black,%
	filecolor = magenta,%
	setpagesize = false,%
	}

\pdfstringdefDisableCommands{%
\renewcommand*{\bm}[1]{#1}%
% any other necessary redefinitions
}
% Include \subsubsection in ToC
\setcounter{tocdepth}{3}

% tabularx
\newcolumntype{C}{>{\centering\arraybackslash}X} %セル内で中央揃え
\newcolumntype{R}{>{\raggedright\arraybackslash}X} %セル内で右揃え
\newcolumntype{L}{>{\raggedleft\arraybackslash}X}

\begin{document}
\section{問題設定及び解法}
無重力場でのスピン衛星の姿勢運動をコンピュータでシミュレートする. シミュレーションのソースコードは
付録にまとめた.

\subsection{条件}
条件は以下のように設定する
\begin{enumerate}
  \item 姿勢表現はQuartanionを用いることとし, 初期条件は $\bm{q} = {(1, 0, 0, 0)}^{\mathrm{T}}$とする.
  \item x,y,z軸はprinciple axisに一致しているとする. $I_x, I_y, I_z$をそれぞれ$1.9,1.6,2.0 [kgm^2]$とする.
  \item y軸周りにノミナルの角速度$\omega_s$ (=17rpm)のスピン角速度があるとする.
  \item 外乱トルク, 制御トルクをそれぞれ$\bm{M_D},\bm{M_C}$とする. 今回のシミュレーションではそれらを0とおく.
  \item Gravity Gradeint その他の外乱トルクは考えず, 重力の影響やエネルギー散逸もないとする.
  \item $\bm{\omega^b}$の初期値は${(0.1,\omega_s +0.1, 0.0)}^{\mathrm{T}}$とする.
\end{enumerate}

\subsection{解くべき方程式}
オイラーの運動方程式は,x,y,zを慣性主軸に取れば,
\begin{equation}
  \bm{M_D} + \bm{M_C}
  =
  \begin{bmatrix}
    I_x\dot{\omega_x} - (I_y - I_z)\omega_y \omega_z \\
    I_y\dot{\omega_y} - (I_z - I_x)\omega_z \omega_x \\
    I_z\dot{\omega_z} - (I_x - I_y)\omega_x \omega_y
  \end{bmatrix}
\end{equation}
今回は, $\bm{M_D} = \bm{M_C} = 0$であるから, ${\bm{\omega^b}}$について解くべき方程式は,
\begin{equation}
  \dot{\bm{\omega^b}} = \frac{d}{dt}
  \begin{bmatrix}
    \omega_x \\
    \omega_y \\
    \omega_z
  \end{bmatrix}
  =
  \begin{bmatrix}
    \displaystyle \frac{I_y - I_z}{I_x} \omega_y \omega_z \\
    \displaystyle \frac{I_z - I_x}{I_y} \omega_z \omega_x \\
    \displaystyle \frac{I_x - I_y}{I_z} \omega_x \omega_y
  \end{bmatrix}
\end{equation}
となる. よって, Quartanion $\bm{q}$について解くべき方程式は,
\begin{equation}
  \dot{\bm{q}} = \frac{d}{dt}
  \begin{bmatrix}
    q_0 \\
    q_1 \\
    q_2 \\
    q_3 \\
  \end{bmatrix}
  = \frac{1}{2}
  \begin{bmatrix}
    -q_1 & -q_2 & -q_3 \\
    q_0 & -q_3 & q_2 \\
    q_3 & q_0 & -q_1 \\
    -q_2 & q_1 & q_0
  \end{bmatrix}
  \begin{bmatrix}
    \omega_x \\
    \omega_y \\
    \omega_z
  \end{bmatrix}
\end{equation}

\subsubsection{数値解法}
式2と式3を4次のルンゲクッタ法で数値積分することにより,
$\bm{\omega^b}$,$\bm{q}$の時間変化を求める.



\end{document}
