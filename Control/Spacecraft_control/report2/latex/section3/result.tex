\documentclass[class=article, crop=false, dvipdfmx, fleqn]{standalone}
\title{航空機設計法第一 \\
レポート課題3 \ 機体三面図(初期案)}
\author{学籍番号 03-170313 飯山 敬大\\
        }
\date{\today}

% packages and libraries
\usepackage[utf8]{inputenc}				%fonts
\usepackage[ipaex]{pxchfon}
\usepackage{pifont}
\usepackage{mathtools, amssymb, mathrsfs, bbm,nccmath}	%math
\usepackage{siunitx, physics}
\usepackage[table]{xcolor}				%colors
\usepackage{tabularx}
\usepackage[dvipdfmx]{graphicx}					%figures
\usepackage{subcaption, wrapfig}
\usepackage{tikz}
\usetikzlibrary{calc, patterns, decorations, angles, calendar, backgrounds, shadows, mindmap}
\usepackage{tcolorbox}					%tables
\usepackage{longtable, float, multirow, array, listliketab, enumitem, tabularx}
\usepackage{listings}					%listings
\usepackage{comment}
\usepackage{hyperref}					%URL, link
\usepackage{url}
\usepackage{pxjahyper}
\usepackage{overcite}					%setting of citation
\usepackage{pxrubrica}					%rubi
\usepackage{fancyhdr, lastpage}			%pagelayout
\usepackage{import, grffile}			%file management
\usepackage{standalone}
\usepackage{bm}
\usepackage{empheq}
\usepackage{pdfpages}
\usepackage{multicol}
% set up for siunitx
\sisetup{%
	%detect-family = true,
	detect-inline-family = math,
	detect-weight = true,
	detect-inline-weight = math,
    %input-product = *,
    quotient-mode = fraction,
	fraction-function = \frac,
	inter-unit-product = \ensuremath{\hspace{-1.5pt}\cdot\hspace{-1.5pt}},
	per-mode = symbol,
	product-units = single,
	}

% setting of line skip
\setlength{\lineskiplimit}{6pt}
\setlength{\lineskip}{6pt}

% setting of indent
\setlength{\parindent}{1zw}
\setlength{\mathindent}{5zw}

% change cite form
\renewcommand{\citeform}[1]{[#1]}

% number equations only when they are referred to in the text
\mathtoolsset{showonlyrefs=true}
%\graphicspath{{images/}{../images/}}

% set up for hyperref
\hypersetup{%
	bookmarksnumbered = true,%
	hidelinks,%
	colorlinks = true,%
	linkcolor = black,%
	urlcolor = cyan,%
	citecolor = black,%
	filecolor = magenta,%
	setpagesize = false,%
	}

\pdfstringdefDisableCommands{%
\renewcommand*{\bm}[1]{#1}%
% any other necessary redefinitions
}
% Include \subsubsection in ToC
\setcounter{tocdepth}{3}

% tabularx
\newcolumntype{C}{>{\centering\arraybackslash}X} %セル内で中央揃え
\newcolumntype{R}{>{\raggedright\arraybackslash}X} %セル内で右揃え
\newcolumntype{L}{>{\raggedleft\arraybackslash}X}

\begin{document}
\section{結果}
Pの初期条件は,
\begin{equation}
  \bm{P} = \begin{bmatrix}
  \sigma_v & 0 & 0 & 0 & 0 & 0 & 0 \\
  0 & \sigma_v & 0 & 0 & 0 & 0 & 0 \\
  0 & 0 & \sigma_v & 0 & 0 & 0 & 0 \\
  0 & 0 & 0 & \sigma_v & 0 & 0 & 0 \\
  0 & 0 & 0 & 0 & \sigma_w & 0 & 0 \\
  0 & 0 & 0 & 0 & 0 & \sigma_w & 0 \\
  0 & 0 & 0 & 0 & 0 & 0 & \sigma_w
\end{bmatrix}
\end{equation}
とした.結果は以下のようになった.

\begin{figure}[H]
 \begin{minipage}{0.5\hsize}
  \begin{center}
   \includegraphics[width=65mm]{../../../fig/R2qt0v1.png}
  \end{center}
 \end{minipage}
 \begin{minipage}{0.5\hsize}
  \begin{center}
   \includegraphics[width=65mm]{../../../fig/R2qt1v1.png}
  \end{center}
 \end{minipage}
\end{figure}

\vspace{-1cm}

\begin{figure}[H]
 \begin{minipage}{0.5\hsize}
  \begin{center}
   \includegraphics[width=65mm]{../../../fig/R2pmatrix0v1.png}
  \end{center}
  %\caption{qt0}
 \end{minipage}
 \begin{minipage}{0.5\hsize}
  \begin{center}
   \includegraphics[width=65mm]{../../../fig/R2pmatrix0v1.png}
  \end{center}
  %\caption{qt1}
 \end{minipage}
\end{figure}

\begin{figure}[H]
 \begin{minipage}{0.5\hsize}
  \begin{center}
   \includegraphics[width=65mm]{../../../fig/R2qt2v1.png}
  \end{center}
 \end{minipage}
 \begin{minipage}{0.5\hsize}
  \begin{center}
   \includegraphics[width=65mm]{../../../fig/R2qt3v1.png}
  \end{center}
 \end{minipage}
\end{figure}

\vspace{-1cm}

\begin{figure}[H]
 \begin{minipage}{0.5\hsize}
  \begin{center}
   \includegraphics[width=65mm]{../../../fig/R2pmatrix2v1.png}
  \end{center}
  %\caption{qt2}
 \end{minipage}
 \begin{minipage}{0.5\hsize}
  \begin{center}
   \includegraphics[width=65mm]{../../../fig/R2pmatrix3v1.png}
  \end{center}
  %\caption{qt3}
 \end{minipage}
\end{figure}

\vspace{2cm}

\begin{figure}[H]
 \begin{minipage}{0.5\hsize}
  \begin{center}
   \includegraphics[width=65mm]{../../../fig/R2omega1v1.png}
  \end{center}
 \end{minipage}
 \begin{minipage}{0.5\hsize}
  \begin{center}
   \includegraphics[width=65mm]{../../../fig/R2omega2v1.png}
  \end{center}
 \end{minipage}
\end{figure}

\vspace{-1cm}

\begin{figure}[H]
 \begin{minipage}{0.5\hsize}
  \begin{center}
   \includegraphics[width=65mm]{../../../fig/R2pmatrix4v1.png}
  \end{center}
  %\caption{$\omega_x$}
 \end{minipage}
 \begin{minipage}{0.5\hsize}
  \begin{center}
   \includegraphics[width=65mm]{../../../fig/R2pmatrix5v1.png}
  \end{center}
  %\caption{$\omega_y$}
 \end{minipage}
\end{figure}

\begin{figure}[H]
 \begin{minipage}{0.5\hsize}
  \begin{center}
   \includegraphics[width=65mm]{../../../fig/R2omega3v1.png}
  \end{center}
 \end{minipage}
 \begin{minipage}{0.5\hsize}
 \end{minipage}
\end{figure}

\vspace{-1cm}

\begin{figure}[H]
 \begin{minipage}{0.5\hsize}
  \begin{center}
   \includegraphics[width=65mm]{../../../fig/R2pmatrix6v1.png}
  \end{center}
  %\caption{$\omega_z$}
 \end{minipage}
 \begin{minipage}{0.5\hsize}
 \end{minipage}
\end{figure}

推定値が真値に漸近し,また真値とのずれ$\Delta x$も$\sqrt{P_{ii}}$が減少するに従って,
減少していっていることが分かる.

\end{document}
