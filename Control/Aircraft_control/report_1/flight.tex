\documentclass[15pt,uplatex,dvipdfmx]{jsarticle}
\usepackage{tabularx}
\usepackage[hmargin=2cm,vmargin=2cm]{geometry}
\usepackage{amsmath}
\usepackage{listings}
\usepackage{float}
\usepackage[dvips]{graphicx}

\title{航空機力学第二 \, 課題1}
\author{航空宇宙工学科3年 \\ 学生記番号:03-170313 \\ 飯山敬大}
\date{2017/10/1}

\begin{document}

\parindent = 0pt

\maketitle

諸元は以下の通りとする(以下の迎角の単位はrad[s]) \\
質量 $m=7500[kg]$ \\
翼面積 $S=27.9[m^2]$ \\
揚力係数 $C_L = C_{L\alpha}\alpha \ (C_{L\alpha} =4.30)$ \\
抗力係数 $C_D = C_{D0} + K{\alpha}^2 \ (C_{D0} = 0.0548 , K = 3.02)$ \\
また各種記号については、プリントにならうものとする。

\section{}
この航空機が高度5000m(空気密度$\rho=0.74[kg/m^3]$),速度U=180[m/s]で水平定常直線飛行を行うときの
迎角$\alpha[deg]$と推力T[N]を求める \\
水平定常直線飛行を行うためには、\\
\begin{equation}
    \dot{z_e} = -U \sin{\gamma} =0 \\
\end{equation}

\begin{equation}
    \dot{U} = \frac{-D + T\cos{\alpha}}{m} - g\sin{\gamma} =0 \\
\end{equation}

\begin{equation}
    \dot{\gamma} = \frac{1}{U}(\frac{L+T\sin{\alpha}}{m}\cos{\Phi}-g\cos{\gamma}) = 0 \\
\end{equation}

の3式を満たす必要がある。ここで(1)より$\gamma = 0$であり、また$\alpha$が
微小であると仮定すれば、$\cos{\alpha} = 1 - \frac{1}{2} {\alpha}^2$ ,\, $\sin{\alpha} = \alpha$
と置けるので、(2),(3)より
\begin{equation}
   \dot{U} = - q_{\infty}S(C_{D0} + K{\alpha}^2) + T(1 - \frac{1}{2} \alpha^2) = 0 \\
\end{equation}
\begin{equation}
  \dot{\gamma} = - q_{\infty}S C_{L\alpha} \alpha + T\alpha - mg = 0\\
\end{equation}
ただし、$ q_{\infty}= \frac{1}{2}\rho U^2 $である。 \\
これを解くと、\\
\begin{equation}
  \begin{cases}
    \alpha = 2.8879 \, [deg] \\
    T = 20921 \, [N] \\
  \end{cases}
\end{equation}
となる。

\section{}
航空機が1の状態で0-20[s]の間飛行し、以下の式の様なロール角制御を行うときの航空機の挙動を調べる
\begin{equation}
  \Phi(t) =
  \begin{cases}
      0 & (0<t<20) \\
      {\Phi}_0 \sin{\omega(t-20[s])} & (20≤t≤60) \\
      0 & (60<t<80) \\
  \end{cases}
\end{equation}

航空機についての3自由度方程式の6つの微分方程式についてルンゲクッタ法を用いて、航空機の飛行経路、速度、姿勢
を計算する。\\
計算に用いたソースコードは以下の通りである \\

\begin{lstlisting}[basicstyle=\ttfamily\footnotesize, frame=single]

%matplotlib inline
from mpl_toolkits.mplot3d import Axes3D
import matplotlib.pyplot as plt
import numpy as np
import sympy as sym

#parameter
m = 7500 #[kg]
s = 27.9 #[m^2]
cl_alpha = 4.30
cd_0 = 0.0548
k = 3.02
rho = 0.74 #[kg/m^3]
g = 9.806 #[m_s^2]
alpha = 0 #radian
thrust = 0

#array for results
t_array = []
u_array = []
gamma_array = []
psi_array = []
phi_array = []
xe_array = []
ye_array = []
ze_array = []

pi = 3.141592
rad = pi/180

def cal_alpha_thrust():
    global m,s,rho,g,cl_alpha,cd_0,k,alpha,thrust,pi,rad
    rad = pi / 180

    (alpha_s,thrust_s) = sym.symbols("alpha_s thrust_s")
    u = 180
    q = 0.5 * rho * u**2

    f1 = - q*s*(cd_0 + k*alpha_s**2) + thrust_s * (1 - alpha_s**2/2)
    f2 = q * s * cl_alpha * alpha_s + thrust_s*alpha_s - m*g
    res = sym.solve([f1,f2],[alpha_s,thrust_s])
    alpha = res[0][0]
    thrust = res[0][1]
    print("alpha=",alpha/rad,"[deg]")
    print("Thrust=",thrust,"[N]")

#roll
def phi(time):
    global m,s,rho,g,cl_alpha,cd_0,k,alpha,thrust,pi,rad
    if time < 20:
        return 0
    elif time < 60:
        omega = 2 * pi / 40
        rad = pi/180
        return 30 * rad * sym.sin(omega*(time-20))
    else:
        return 0

#velocity
def diff_u(t,u,gamma,psi):
    global m,s,rho,g,cl_alpha,cd_0,k,alpha,thrust,pi,rad
    q = 0.5 * rho * u**2
    drag = q * s * (cd_0 + k*alpha**2)
    diff_u = (-drag + thrust*sym.cos(alpha))/m - g * sym.sin(gamma)
    return diff_u

#route angle
def diff_gamma(t,u,gamma,psi):
    global m,s,rho,g,cl_alpha,cd_0,k,alpha,thrust,pi,rad
    if t < 20:
        return 0
    global cl_alpha,alpha
    q = 0.5 * rho * u**2
    lift = q * s * cl_alpha * alpha
    diff_gamma = ((lift + thrust*sym.sin(alpha))* sym.cos(phi(t))/m - g*sym.cos(gamma)) / u
    return diff_gamma

#yaw
def diff_psi(t,u,gamma,psi):
    global m,s,rho,g,cl_alpha,cd_0,k,alpha,thrust,pi,rad
    q = 0.5 * rho * u**2
    lift = q * s * cl_alpha * alpha
    diff_psi = ((lift + thrust*sym.sin(alpha))/m) * (sym.sin(phi(t))/sym.cos(gamma)) / u
    return diff_psi

def diff_xe(t,u,gamma,psi):
    diff_xe = u * sym.cos(gamma) * sym.cos(psi)
    return diff_xe

def diff_ye(t,u,gamma,psi):
    diff_ye = u * sym.cos(gamma) * sym.sin(psi)
    return diff_ye

def diff_ze(t,u,gamma,psi):
    diff_ze = -u * sym.sin(gamma)
    return diff_ze

def cal_flightroute():
    global m,s,u,rho,g,cl_alpha,cd_0,k,alpha,thrust,pi,rad
    global xe_array,ye_array,ze_array,t_array,u_array,gamma_array,psi_array,phi_array

    #initialize
    gamma = 0
    psi = 0 #yaw
    phi_data=0 #roll
    t=0
    dt = 0.5
    endt = 80.0
    xe = 0
    ye = 0
    ze = -5000

    #array for results
    t_array = [t]
    u_array = [u]
    gamma_array = [gamma]
    psi_array = [psi]
    phi_array = [phi_data]
    xe_array = [xe]
    ye_array = [ye]
    ze_array = [-ze]

    #replace names
    x = u
    y = gamma
    z = psi

    #array for rungekutta method
    k0 = [0,0,0,0,0,0]
    k1 = [0,0,0,0,0,0]
    k2 = [0,0,0,0,0,0]
    k3 = [0,0,0,0,0,0]

    while t < endt:
        k0[0]=dt*diff_u(t,x,y,z)
        k0[1]=dt*diff_gamma(t,x,y,z)
        k0[2]=dt*diff_psi(t,x,y,z)
        k0[3]=dt*diff_xe(t,x,y,z)
        k0[4]=dt*diff_ye(t,x,y,z)
        k0[5]=dt*diff_ze(t,x,y,z)

        k1[0]=dt*diff_u(t+dt/2.0,x+k0[0]/2.0,y+k0[1]/2.0,z+k0[2]/2.0)
        k1[1]=dt*diff_gamma(t+dt/2.0,x+k0[0]/2.0,y+k0[1]/2.0,z+k0[2]/2.0)
        k1[2]=dt*diff_psi(t+dt/2.0,x+k0[0]/2.0,y+k0[1]/2.0,z+k0[2]/2.0)
        k1[3]=dt*diff_xe(t+dt/2.0,x+k0[0]/2.0,y+k0[1]/2.0,z+k0[2]/2.0)
        k1[4]=dt*diff_ye(t+dt/2.0,x+k0[0]/2.0,y+k0[1]/2.0,z+k0[2]/2.0)
        k1[5]=dt*diff_ze(t+dt/2.0,x+k0[0]/2.0,y+k0[1]/2.0,z+k0[2]/2.0)

        k2[0]=dt*diff_u(t+dt/2.0,x+k1[0]/2.0,y+k1[1]/2.0,z+k1[2]/2.0)
        k2[1]=dt*diff_gamma(t+dt/2.0,x+k1[0]/2.0,y+k1[1]/2.0,z+k1[2]/2.0)
        k2[2]=dt*diff_psi(t+dt/2.0,x+k1[0]/2.0,y+k1[1]/2.0,z+k1[2]/2.0)
        k2[3]=dt*diff_xe(t+dt/2.0,x+k1[0]/2.0,y+k1[1]/2.0,z+k1[2]/2.0)
        k2[4]=dt*diff_ye(t+dt/2.0,x+k1[0]/2.0,y+k1[1]/2.0,z+k1[2]/2.0)
        k2[5]=dt*diff_ze(t+dt/2.0,x+k1[0]/2.0,y+k1[1]/2.0,z+k1[2]/2.0)

        k3[0]=dt*diff_u(t+dt,x+k2[0],y+k2[1],z+k2[2])
        k3[1]=dt*diff_gamma(t+dt,x+k2[0],y+k2[1],z+k2[2])
        k3[2]=dt*diff_psi(t+dt,x+k2[0],y+k2[1],z+k2[2])
        k3[3]=dt*diff_xe(t+dt,x+k2[0],y+k2[1],z+k2[2])
        k3[4]=dt*diff_ye(t+dt,x+k2[0],y+k2[1],z+k2[2])
        k3[5]=dt*diff_ze(t+dt,x+k2[0],y+k2[1],z+k2[2])

        x=x+(k0[0]+2.0*k1[0]+2.0*k2[0]+k3[0])/6.0
        y=y+(k0[1]+2.0*k1[1]+2.0*k2[1]+k3[1])/6.0
        z=z+(k0[2]+2.0*k1[2]+2.0*k2[2]+k3[2])/6.0
        xe=xe+(k0[3]+2.0*k1[3]+2.0*k2[3]+k3[3])/6.0
        ye=ye+(k0[4]+2.0*k1[4]+2.0*k2[4]+k3[4])/6.0
        ze=ze+(k0[5]+2.0*k1[5]+2.0*k2[5]+k3[5])/6.0

        t_array.append(t)
        u_array.append(x)
        gamma_array.append(y/rad)
        psi_array.append(z/rad)
        phi_data = phi(t)
        phi_array.append(phi_data/rad)
        xe_array.append(xe)
        ye_array.append(ye)
        ze_array.append(-ze)
        t += dt

\end{lstlisting} \newpage

結果のグラフは以下の様になる \\
機体のルート及び速度は以下の様になる \\

\begin{figure}[H]
\begin{center}
  \includegraphics[scale=0.8]{flightroute.png}
\end{center}
  \caption{飛行経路}
\end{figure}

\begin{figure}[H]
\begin{center}
  \includegraphics[scale=0.8]{U.png}
\end{center}
  \caption{飛行速度U(時間変化)}
\end{figure} \newpage

機体の位置の時間変化は以下のグラフの様になる \\

\begin{figure}[H]
\begin{center}
  \includegraphics[scale=0.6]{x_e.png}
\end{center}
  \caption{$x_e$(時間変化)}
\end{figure}

\begin{figure}[H]
\begin{center}
  \includegraphics[scale=0.6]{y_e.png}
\end{center}
  \caption{$y_e$(時間変化)}
\end{figure}

\begin{figure}[H]
\begin{center}
  \includegraphics[scale=0.6]{altitude.png}
\end{center}
  \caption{飛行高度}
\end{figure} \newpage

機体の姿勢は以下のグラフの様になる \\

\begin{figure}[H]
\begin{center}
  \includegraphics[scale=0.6]{gamma.png}
\end{center}
  \caption{経路角$\gamma$}
\end{figure}

\begin{figure}[H]
\begin{center}
  \includegraphics[scale=0.6]{psi.png}
\end{center}
  \caption{ヨー$\Psi$}
\end{figure}

\begin{figure}[H]
\begin{center}
  \includegraphics[scale=0.6]{phi.png}
\end{center}
  \caption{ロール$\Phi$}
\end{figure}


\section{}
最後に、上記の様なロール角制御を行った場合に、速度と高度を一定に保つために行うべき迎角と推力の
制御について考える \\
高度を一定に保つためには、\\
\begin{equation}
    \dot{z_e} = -U \sin{\gamma} =0 \\
\end{equation}

とならなくてはならないから、常に$\gamma=0$とならなくてはならない。\\
よって、速度と高度を一定に保つためには、

\begin{equation}
    \dot{U} = \frac{-D + T\cos{\alpha}}{m} - g\sin{\gamma} =0 \\
\end{equation}

\begin{equation}
    \dot{\gamma} = \frac{1}{U}(\frac{L+T\sin{\alpha}}{m}\cos{\Phi}-g\cos{\gamma}) = 0 \\
\end{equation}

の2式が任意の時刻tにおいて成り立つ様に$\alpha$とTを制御する必要がある。ここで(1)より$\gamma = 0$であり、また$\alpha$が
微小であると仮定すれば、$\cos{\alpha} = 1 - \frac{1}{2} {\alpha}^2$ ,\, $\sin{\alpha} = \alpha$
と置けるので、(2),(3)より各時刻tについて、以下の2式を満たす$\alpha$とTを求めて行けば良い。
\begin{equation}
   \dot{U} = - q_{\infty}S(C_{D0} + K{\alpha}^2) + T(1 - \frac{1}{2} \alpha^2) = 0 \\
\end{equation}
\begin{equation}
  \dot{\gamma} = (- q_{\infty}S C_{L\alpha} \alpha + T\alpha)\cos{\Phi(t)} - mg = 0\\
\end{equation}
ただし、$ q_{\infty}= \frac{1}{2}\rho U^2 $である。 \\
計算には以下のソースコードを用いた \\

\begin{lstlisting}[basicstyle=\ttfamily\footnotesize, frame=single]
def cal_stable():
    global m,s,u,rho,g,cl_alpha,cd_0,k,pi
    stable_alpha = []
    stable_thrust = []
    stable_t = []
    t = 0
    dt = 1
    q = 0.5 * rho * u**2
    (alpha_s,thrust_s) = sym.symbols("alpha_s thrust_s")
    while t < 80:
        f1 = - q*s*(cd_0 + k*alpha_s**2) + thrust_s * (1 - alpha_s**2/2)
        f2 = (q * s * cl_alpha * alpha_s + thrust_s*alpha_s)* sym.cos(phi(t)) - m*g
        res = sym.solve([f1,f2],[alpha_s,thrust_s])
        alpha = res[0][0]
        thrust = res[0][1]
        stable_t.append(t)
        stable_alpha.append(alpha/rad)
        stable_thrust.append(thrust)
        t += dt
\end{lstlisting} \newpage

結果は以下の図の様になる

\begin{figure}[H]
\begin{center}
  \includegraphics[scale=0.6]{stable_alpha.png}
\end{center}
  \caption{迎角$\alpha$[deg]}
\end{figure}

\begin{figure}[H]
\begin{center}
  \includegraphics[scale=0.6]{stable_thrust.png}
\end{center}
  \caption{推力T[N]}
\end{figure}


\end{document}
